
\textbf{Step 3:} Generating the shell script to run the Matlab code. \\

\noindent
In this step we are writing a shell script to run the Matlab code. Note that shell script does not control where Matlab is supposed to read the data or write the data, it just runs the code. In \texttt{Function\_script} folder we create a file and name it \texttt{average\_grade.sh}. Since we are in the \texttt{Function\_script} folder and both Matlab and Shell script are inside this folder, at this point we don't need to handle the path.  The  \texttt{average\_grade.sh} is according below:\\
 
 \begin{mdframed}[hidealllines=true,backgroundcolor=gray!20]
 \begin{singlespace}
 \fontsize{10pt}{1pt}
\texttt{
\\
\noindent
{ \color{matlab_green} \#!/bin/sh} \\
matlab -nodisplay -nosplash -r {\color{red}"matlab\_st\_grade,quit"}\\
 }
 \end{singlespace}
\end{mdframed}

\noindent
First line set the shell environment. At the second line we call Matlab application (make sure you exported the Matlab path as it mentioned in Matlab section), then ask the application to run  the \texttt{matlab\_st\_grade.m} script (does not need .m) and quit the program.  So far everything is set up. Here is the summary of what we have done so far. \\

\begin{itemize}

\item{We have a Matlab code which does some processing. However, it reads input and output from the same directory. \tab \ We need to modify it to read data from and write the results into \texttt{Output\_files}.}
\item{The Matlab code reads file by exact file name. \tab \ It should read the file by the name that we assign to the file.}
\item{We have a shell script that run the Matlab. \tab \ We need to modify the script to also provide the input file and take care of the output file of the Matlab.}
\end{itemize}



