\section{Basic Shell Commands}

Here is the list of important shell commands for the project. You may need other commands, this depends on the complication of your project. \\


\begin{table}[h]

\begin{tabular}{ m{2cm}  m{12cm} }

\hline
 \textbf{Command}          &  \textbf{Description}\\ 
\hline
 \textbf{mkdir}          & make a directory \\ 
 \textbf{ls}                & list directory contents \\ 
 \textbf{cd}               & change the directory\\
 \textbf{pwd}            & return working directory name \\
 \textbf{cp}               & copy files \\ 
 \textbf{mv}              & move files \\ 
 \textbf{rm}               & remove directory entries \\ 
 \textbf{cat}              & concatenate and print files \\    
 \textbf{head}           & display first lines of a file\\
 \textbf{tail}              & display the last part of a file\\
 \textbf{grep}            & file pattern searcher\\   
 \textbf{echo}           & write arguments to the standard output\\
 \textbf{paste}          & Merge corresponding or subsequent lines of files\\
 \textbf{sed}             & stream editor\\   
 \textbf{chmod}        & change file modes or access control lists\\ 
 \hline
 \end{tabular}
\label{tab:b_k_m_param}
\end{table}

\noindent
The basic shell commands are simple and easy to use. The beauty of the shell script is allowing the user to combine the commands together for complicated 
processes. Combining commands or passing data between commands happens by understanding the concept of "pipe, standard input," and "standard output." Generally, all commands print out some results into the screen. Standard input is the source of input data which can come from the keyboard, a file, or a standard output of another program. Sometimes, instead of obtaining the input from the keyboard or from a file, a program can use the output of another program as its input. This is accomplished through the use of a "pipe." The vertical line $"\vert"$  is what is known as pipe. \\
\noindent
Here is an example of using the pipe ($\vert$) to pass data among three different commands. The example is related to seismic data. Much of seismic data comes with a header. Generally, the header is a summary of information about the data (e.g. length, station location, components, ...). Following is the part of data from Southern California Earthquake Data Center at Caltech. The data format is COSMOS V1. The time increment of the data is indicated by "dt=" and it is located at the end of the line. \\



\begin{mdframed}[hidealllines=true,backgroundcolor=gray!20]
\begin{singlespace}
\fontsize{8pt}{1pt}
\textrm{
...\\
   3 Comment lines(s) follow, each starting with a '$\vert$':     \\                    
$\vert$Station: CI ADO HNE   34.5505   -117.4339 908.00   4.0978e+03 \\                 
$\vert$ Record: start= 2008/07/29,18:41:59.721 len=  338.630 ( 33864 pts) dt= 0.0100  \\
$\vert$ Event:  14383980 2008/07/29,18:42:15.710   33.9530   -117.7613  14.70  5.39   \\
   33864 velocity pts.     approx  338 secs, units=cm/sec (05),Format=(6E12.4)  \\
  5.9990e-03  3.8027e-03 -1.0595e-02 -5.8995e-04  6.7311e-03 -1.5661e-03    \\    
  4.5348e-03  4.2908e-03 -6.6908e-03 -3.2743e-03  8.4394e-03 -2.0542e-03     \\   
 -9.1312e-03  5.5109e-03  6.4871e-03 -3.4591e-04 -1.5661e-03  5.2669e-03   \\     
 ...
}
\end{singlespace}
\end{mdframed}
\noindent
The following script can extract the dt value from the file. First we read the file and pipe it to the \texttt{grep} command to find those lines with "dt". We pipe the result into the \texttt{awk} command and print out the last column (NF = number of fields). 

\begin{mdframed}[hidealllines=true,backgroundcolor=gray!20]
\fontsize{10pt}{1pt}
\texttt{cat SeismicData.V1 | grep "dt=" | awk \color{red}{'\{print \$NF\}'}}
\end{mdframed}



