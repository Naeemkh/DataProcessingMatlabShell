\textbf{Step 5:} Modifying the  \texttt{average\_grade.sh} script to read all data in the \texttt{input\_files} folder \\
\\
\noindent
In simple programs, the \texttt{run.sh} file, commonly runs another script in the \texttt{Functions\_scripts} folder, however, it is good to have it, since it can decrease the complexity for user. If you set up the \texttt{run.sh} properly, the user will only need to execute the run.sh file, without getting too much involve in details about the program. For this project the \texttt{run.sh} file is according below:\\

 \begin{mdframed}[hidealllines=true,backgroundcolor=gray!20]
 \begin{singlespace}
 \fontsize{10pt}{1pt}
\texttt{
\\
\noindent
{ \color{matlab_green}  \#!/bin/sh}\\
{ \color{matlab_green}  \# Program Name: Students Final Grade}\\
{ \color{matlab_green}  \# Written by: Naeem Khoshnevis}\\
{ \color{matlab_green}  \# Last update: Jun 5, 2016}\\
\\
\\
{ \color{matlab_green} \# ----------------------------------------------------------------} \\
{ \color{matlab_green} \# --------------------- Input files format -----------------------} \\
{ \color{matlab_green} \# student.txt:} \\
{ \color{matlab_green} \# 3.5} \\
{ \color{matlab_green} \# 4} \\
{ \color{matlab_green} \# 3.2} \\
{ \color{matlab_green} \# .} \\
\\
\\
{ \color{matlab_green} \# ----------------------------------------------------------------} \\
{ \color{matlab_green} \# --------------------- Change the directory ---------------------} \\
\\
cd \texttt{Functions\_scripts}\\
\\
{ \color{matlab_green} \# ----------------------------------------------------------------} \\
{ \color{matlab_green} \# --------------------- Run the program --------------------------} \\
\\
./average\_grade.sh ../Input\_files\\
 }
 \end{singlespace}
\end{mdframed}
\noindent
We set up the \texttt{run.sh} file. You can put as many details about the program (as comments) in this file, as you want. Now we need to modify the \texttt{average\_grade.sh} script. The file should be able to run all the input files (based on a special pattern) from the \texttt{Input\_files} folder and process them. In this project we assume all input files are \texttt{.txt} files. Here is the modified version of the \texttt{average\_grade.sh} script.





 
\begin{mdframed}[hidealllines=true,backgroundcolor=gray!20]
\begin{singlespace}
\fontsize{10pt}{1pt}
\texttt{
\\
{ \color{matlab_green} \#!/bin/sh}\\
{ \color{matlab_green} \# This function call the Matlab script to take the average of students grades.}\\
{ \color{matlab_green} \# results will be at the output\_files folder}\\
\\
{ \color{matlab_green} \# go in to the first variable after the ./xx.sh command.}\\
{ \color{matlab_green} \# In this case is ../Inpute\_files.}\\
\\
cd \$1\\
\\
{\color{for_pink}for} i {\color{for_pink}in} \$(ls *.txt)\\
\\
{\color{for_pink}do}\\
\\
current\_path={\color{red}`pwd`}\\
script\_path={\color{red}`echo \$current\_path| sed -e 's/Input\_files/Functions\_scripts/g'`}\\
\\
cat \$my\_file > ../Output/temp\_grade\_file.txt\\
\\
matlab -nodisplay -nosplash -r   \textbackslash \\
{\color{red}"try,run \$script\_path/matlab\_st\_grade\_temp,catch,end,quit"}\\
\\
number\_of\_grades={\color{red}`cat ../Output\_files/temp\_grade\_mat.txt | awk '{print \$1}'`}\\
average={\color{red}`cat ../Output/temp\_grade\_mat.txt | awk '{print \$2}'`}\\
echo \$my\_file {\color{red}'Number of grades:'}  \$number\_of\_grades \textbackslash \\
\phantom{x}\hspace{14ex} {\color{red}'Average score:'} \  \$average >> ../Output/Final\_grade.txt \\
\\
{\color{for_pink}done}\\
\\
rm ../Output\_files/temp*  \\  
 }
\end{singlespace}
\end{mdframed}
\noindent
The package is now complete. You may put as many students' grade files in the \texttt{Input\_files} folder and get the results at the \texttt{Output\_files} folder. The \texttt{Final\_grade.txt} file for five students could be similar to the following:

\begin{mdframed}[hidealllines=true,backgroundcolor=gray!20]
\begin{singlespace}
\fontsize{10pt}{1pt}
\texttt{
\\
> cat Final\_grade.txt \\
Amy.txt \phantom{x}\hspace{3ex}     Number of grades:  31  Average score:  3.306 \\
David.txt \phantom{x}\hspace{1ex}   Number of grades:  19  Average score:  3.387 \\
John.txt \phantom{x}\hspace{2ex}    Number of grades:  28  Average score:  3.408 \\
Matt.txt \phantom{x}\hspace{2ex}    Number of grades:  22  Average score:  3.334 \\
Rob.txt  \phantom{x}\hspace{3ex}     Number of grades:  22  Average score:  3.334 \\
 }
\end{singlespace}
\end{mdframed}
\noindent
The package processes the data properly. However, there is a potential problem. If some data is not as what we assumed, we will never be informed about it. In the next step, we modify the code to be able to report the potential problems in the data.\\


